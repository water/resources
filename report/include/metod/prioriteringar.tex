\section{Prioriteringar}

Projektet delas upp i tre prioritetsnivåer, där funktioner på den första nivån är absoluta krav som ska realiseras under den första iterationscykeln.

\subsection{Första prioritet}

\begin{itemize}
\item Som en student ska man kunna registrera sig på Water
\item Som en student ska man kunna lämna in sin laboration via versionshanterare och webbgränssnitt
\item Som en student ska man kunna få sin laboration rättad
\item Som en handledare ska man kunna rätta laborationer
\item Som en examinator ska man kunna skapa en kurs
\item Som en examinator ska man kunna skapa en laboration
\item Som en examinator ska man kunna modifiera sina befintliga kurser och laborationer
\item Som en examinator ska man kunna skriva ut LADOK-vänliga listor
\item Som en examinator ska man kunna delegera vissa administrativa rättigheter till handledare
\end{itemize}

\subsection{Andra prioritet}

\begin{itemize}
\item Som en student ska man kunna ha en interaktiv dialog med sin handledare med hjälp av systemet
\item Som en handledare ska man med hjälp av systemet kunna få stöd till rättning i form av enkla valideringar
\item Som en examinator ska man få tillgång till aktuell statistik
\end{itemize}

\subsection{Tredje prioritet}

\begin{itemize}
\item Undersöka möjligheten att skriva betyg direkt till LADOK
\item Undersöka möjligheten till automatisk plagiatkontroll
\item Undersöka möjligheten att validera inlämningar med hjälp av unit/acceptance tests
\item Undersöka anonym rättning
\end{itemize}
