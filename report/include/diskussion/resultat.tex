\section{Resultat}
Resultatet av projektet är en mogen backend med ett vältestat modell-lager. Se kodtäckningsbilaga för mer information. Alla subsystem har implementerats, men med varierande grader av fullständighet. Systemet som helhet följer konventioner för de ramverk som använts vilket gör det enkelt för framtida vidareutveckling och underhåll.

Inlämning via Git är delvis funktionellt. Inlämning via webbgränssittet är i stort sett klart.

Webbgränssnittet för studenter är i stort sett klart. För handledare finns många viktiga funktioner implementerade.

Kommentarer har blivit implementerade för inlämningar.

Examinatorns och systemadministratörens gränssnitt har inte implementerats än.

Implementationen av resultatlistor som är kompatibla med LADOK är i stort sett klart, det återstår att koppla samman de olika metoderna och skriva tester. En färdig mall \ref{appendix:ladok} finns att tillgå, den har utformats med hjälp av studieadministratörer på Chalmers.\footnote{Intervju med Anna Södergård, Studieadministratör den 20 februari 2012.}

\subsection{Implementerad funktionalitet}
Här presenteras funktionalitet som är implementerad när rapporten skrivs.
\subsubsection{Funktioner}
\begin{itemize}
	\item Webbgränssnitt
	\begin{itemize}
		\item Dashboard – visar viktig information om kurser, laborationer och inlämningar, implementerade för:
		\begin{itemize}
			\item Studenter
			\item Handledare
		\end{itemize}
		\item Bläddra i filsystem
		\item Rendering av källkod.
		\item Inlämningar
		\begin{itemize}
			\item Ladda upp en eller flera filer på samma gång
			\item Radera filer och mappar
			\item Skapa mappar
			\item Göra en inlämning, uppdatera en inlämning om rättningsprocessen inte har inletts och deadline inte har förlupit
		\end{itemize}
		\item Grupphantering
		\begin{itemize}
			\item Skapa grupp i kurs
            \item Knyta grupp till laboration
            \item Knyta student till grupp med hjälp av inbjudningskod
		\end{itemize}
		\item Rättning
		\begin{itemize}
			\item Översikt över alla kurser med inlämningar som har kopplats till handledaren
            \item Lägga till handledares anteckning till Submission
            \item Lägga till inledande kommentar till Submission
            \item Godkänna eller underkänna en Submission
            \item Studenter och handledare kan lägga till ytterligare kommentarer på en Submission
		\end{itemize}
	\end{itemize}
\end{itemize}

\subsubsection{Subsystem}

\begin{itemize}
  \item PostgreSQL-databas
  \item Ruby on Rails-applikation
  \item Faye – WebSocket-server
  \item Javascriptapplikationer
  \item Köhanterare med workers
  \item Grack – SmartHTTP-server
\end{itemize}

Planerad funktionalitet som inte implementerades

\begin{itemize}
  \item Validering av inlämningsuppgifter
  \item Noggrannare hantering av deadlines och hur de relaterar till inlämningar.
  \item Administration  
  \begin{itemize}
    \item Skapa och hantera kurser, inlämningsuppgifter och deadlines
    \item Delegering av rättigheter - implementerad i modellen men ej i gränssnittet
  \end{itemize}
  \item Visualisera diffar
  \item Grupphantering
  \begin{itemize}
    \item Hantering av situationer där gruppen vill splittras efter att arbetet har inletts.
  \end{itemize}
\end{itemize}
