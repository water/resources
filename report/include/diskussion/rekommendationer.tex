\section{Vidareutveckling av systemet}

Implementationen är inte klar. Nästa steg för systemet är därför att all kvarvarande planerad funktionalitet implementeras.

Ännu saknas implementation av valideringar, deadlines och administration. Särskilt deadlines och administration är kritiska funktioner för att systemet skall kunna användas. 

\subsection{Administration}

Administration innefattar all form av hantering av kurser och användare. Kurser behöver kunna skapas och rättigheter delegeras av någon administratör. Kursen i sin tur måste administreras av någon, förslagsvis examinatorn, som i sin tur måste kunna skapa, ta bort och redigera laborationer. Examinatorn borde även kunna registrera studenter på kursen och tillsätta handledare med speciella rättigheter. Slutligen skall implementationen av LADOK-kompatibla listor färdigställas.

\subsubsection{Numrering av inlämningsuppgifter}
I skrivande stund numreras inlämningsuppgifter inom en kurs i den ordning som de skapades. Nummer används sedan på många ställen för att referera till inlämningsuppgiften. Detta blir olyckligt om inlämningsuppifterna skapas i fel ordning. Det bör gå att skapa inlämningsuppgifter i en godtycklig ordning och ändra numreringen senare.

\subsection{Deadlines}
I skrivande stund har varje laboration en deadline. Domänmodellen innehåller även stöd för utökade deadlines för specifika grupper. Restriktioner kopplade till deadlines, som att inlämningen bör stängas i vissa situationer, har dock inte implementerats. Ett annat problem som bör hanteras är prioritering av allmänna kontra individuella deadlines. Det bör även gå att villkora deadlines, till exempel så att en andra deadline endast går att använda om inlämning har skett före den första deadlinen.

\subsection{Valideringar}
Enkla automatiska valideringar av formalia som filnamn, filändelser och liknande är inte en nödvändig del av ett fungerande system, men är en potentiellt populär funktion som borde vara lätt att implementera.

\subsection{Diffar}

Möjligheten att snabbt se skillnader mellan två tillstånd i en inlämningsuppgift är en viktig funktion särskilt för handledarna. Detta finns redan inbyggt i Gitorious och borde därför vara lätt att implementera.

\subsection{Testning}

Det är att föredra om ett testdrivet-utvecklingsätt används även vid vidareutveckling.

De nuvarande testerna på modell-nivå bör i vissa fall utökas. Viktigt är testerna på Controller-nivå utökas så det går att snabbt verifiera att förändringar i kodbasen inte stör viktiga funktioner i webbgränssnittet. 


\subsection{Kommentarer}
Förmågan att kommentera andra modeller utöver Submission har inte blivit implementerad. Det kan till exempel vara av intresse för studenter att kunna ställa frågor om en laboration via kommentarer. 

\subsection{Dokumentation}
För att göra systemet användbart för slutanvändarna måste en användarhandbok skrivas. 

För göra det möjligt för andra utvecklare att vidareutveckla systemet måste en mer gedigen dokumentation av kodbasen tas fram.
