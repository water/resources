% Chalmers title page
\begin{titlepage}

\AddToShipoutPicture{\backgroundpic{-4}{56.7}{fig/auxiliary/frontpage}}
\mbox{}
\vfill
\addtolength{\voffset}{2cm}
\begin{flushleft}
	{\noindent {\Huge Water} \\[0.5cm]
	\emph{\Large En ersättning för Fire, baserad på versionshantering} \\[2.8cm]
	
	{\huge JESPER JOSEFSSON    }\\[.8cm]
	{\huge SOFIA LARSSON       }\\[.8cm]
	{\huge LINUS OLEANDER      }\\[.8cm]
	{\huge ARASH ROUHANI-KALLEH}\\[.8cm]
	{\huge PONTUS SAHLBERG     }\\[.8cm]
	{\huge JONAS ÄNGESLEVÄ     }\\[.8cm]
	
	{\Large Department of Computer Science \& Engineering \\
	\textsc{Chalmers University of Technology} \\
	Gothenburg, Sweden 2012 \\
	Bachelor's Thesis 2012:11\\
	} 
	}

\end{flushleft}

\end{titlepage}
\ClearShipoutPicture
% End Chalmers title page

\pagestyle{empty}
\newpage
\clearpage
\mbox{}
\newpage
\clearpage
\thispagestyle{empty}

\begin{abstract}
    Syftet med projektet som rapporten beskriver är att utveckla ett inlämningssystem för laborationer. Systemet ska bygga på versionshantering och ska hantera inlämning via en terminalklient. Inlämning skall även vara möjlig via ett webbgränssnitt.

    Systemets kärna är en webbapplikation som är baserad på open source-plattformen Gitorious, som är implementerad i Ruby on Rails. Systemet har en avancerad webbklient som flitigt nyttjar MVC-strukturerade Javascript-applikationer. Processintensiva arbeten delegeras till ett system av prioritetsköer och workers. WebSocket-protokollet används för asynkron kommunikation mellan webbservern och klienten. 

    Systemet använder sig utav BDD-ramverket RSpec för att få en självdokumenterande, högkvalitativ kodbas.

    Projektets omfattning visade sig vara för stort för tidsrymden av ett kandidatarbete, men resulterade i en mogen backend och ett välutvecklat gränssnitt som hanterar inlämning samt rättning av inlämningsuppgifter.

\end{abstract}
\selectlanguage{english}
  \begin{abstract}
    The purpose of the project described in the report is to develop a system for receiving and processing hand-in assignments. The system should be based on a version control system. The system should be able to receive hand-ins via a version control system command line client. Hand-in should also be possible via a web interface.

    A web application based on the open source platform Gitorious and Ruby on Rails is used to implement the system. The system features an advanced web client which makes extensive use of Javascript applications structured using the MVC design pattern. Process-intensive jobs are referred to a system of priority queues and workers. The WebSocket protocol is used for asynchronous communication between the web server and the client.

    The system uses the BDD-type test framework Rspec in order to make the code base self-documenting and to ensure high quality code.

    The scope of the project was found to be too wide, but the result was a mature backend and a well developed frontend for handing in and reviewing assignments.
  \end{abstract}
\selectlanguage{swedish}

