\chapter{Inledning}

\section{Bakgrund}

\subsection{Inlämningsuppgiftens roll i datavetenskapliga utbildningar}

Inom datavetenskapliga utbildningar är inlämningsuppgifter, ibland kallade laborationer, ett vanligt pedagogiskt verktyg. I regel ställs studenterna inför någon form av problem som de ska lösa genom att programmera. Genom laborationerna får studenterna möjlighet att applicera koncept som tagits upp under föreläsningar och de kan ställas inför praktiska problem som inte berörs av undervisningen i övrigt. 

Antalet studenter på datavetenskapliga kurser kan vara stort vilket gör att den givande institutionen ställs inför ett praktiskt problem – hur hanterar man inlämning och rättning av en stor mängd inlämningsuppgifter på ett effektivt sätt?

Svaret är ett specialiserat system.

\subsection{Nuvarande system vid Chalmers tekniska högskola}

På Chalmers tekniska högskola används ofta Fire, ett webbaserat system som endast hanterar inlämning och rättning av inlämningsuppgifter.
Studenten skapar en användare och ansluter sig sedan till en grupp i Fires webbgränssnitt. Därefter väljer studenten en laboration att ladda upp filer till. När samtliga filer är uppladdade kan studenten välja att lämna in sina filer. 

Handledarna som rättar uppgifterna erbjuds ett gränssnitt för rättning. Administratören kan skapa kurser och delegera rättigheter, som att förändra deadlines, till handledare. Systemet erbjuder dessutom smidig hantering av kurser med studenter från olika universitet.

Fire uppfyller grundförutsättningarna för ett fungerande system och har många uppskattade egenskaper.
Däremot finns det ett antal områden med utrymme för förbättring. Bland annat startar Fire en separat webbserverinstans för varje kurs. Detta leder till vattentäta skott mellan kurser. Studenter är tvungna att registrera sig på nytt för varje ny kurs och länken till webbgränssnittet är olika för varje kurs. Tidigare har antalet webbserverinstanser dessutom begränsats till 64. I nuläget har har denna artificiella gräns höjts, men i längden är det inte praktiskt att skapa en serverinstans per kurs.


Ett annat problem med Fire är att många operationer inte går att ångra eller ändra i ett senare skede. 
I Fire utnyttjas inte heller möjligheten att erbjuda mervärde till användarna. Kommunikationen mellan handledare och studenter är rudimentär – handledarens svar laddas upp via ett webbformulär och studenterna kan bara svara med en ny inlämning. Projektarbeten hämmas av att inlämning av större mängder filer eller mappstrukturer endast är möjligt med tar.gz-arkiv (Gedell, 2006a).

Orsaken till att många problem kvarstår är att Fire saknar dokumentation på kodbasen och databasen. Detta gör det väldigt svårt att vidareutveckla Fire.
Det finns med andra ord mycket som talar för en helt ny lösning för inlämning och hantering av laborationer.

\subsection{Versionshantering}

Versionshantering är ett oumbärligt verktyg för programmerare. Konceptet \emph{branches} gör det lätt att utveckla nya funktioner utan att förstöra nuvarande kodbas. Versionshantering förenklar även arbetet i grupp genom att stödja sammanslagning - \emph{merge} - av olika grenar. Detta uppmuntrar till strukturerade arbetssätt och effektivt samarbete. 

Backup och historik förflyttas från filnivå till funktionsnivå genom att man gör regelbundna \emph{commits}. Dessa upptar dessutom nästan ingen plats eftersom systemen i regel inte sparar hela projektets tillstånd i en commit, utan bara skillnaden till föregående commit – så kallad $\Delta$-data.

I en commit förväntas programmeraren bifoga ett meddelande som beskriver ändringarna som har gjorts. Detta uppmanar fram en form av loggskrivande under utvecklingsprocessen.

Versionshantering underlättar även vid bugghantering, eftersom flera kända buggfria tillstånd ofta finns att hitta i historiken.

I projektgruppen använder vi redan versionhantering för att hantera våra inlämningsuppgifter.

Om ett inlämningssystem utformas så att det bygger på versionshantering skulle systemet kunna erbjuda inlämningar direkt från versionshanteringsklienten. Systemet skulle dessutom kunna utnyttja fördelarna med versionhantering, som att snabbt kunna visa skillnader mellan olika tillstånd i kodbasen.

\section{Syfte}

Huvudsyftet med projektet är att konstruera ett system med webbgränssnitt och
backend som hanterar inlämning och rättning av inlämningsuppgifter. Systemet ska
vara baserat på versionshantering.

\section{Specifika krav och centrala frågeställningar}

För att begränsa projektets omfattning, särskilt under analysfasen, har kandidatgruppen
ställt upp ett antal specifika krav på systemet.
Kandidatgruppen har även formulerat några viktiga frågeställningar som ska undersökas under projektets gång.

\subsection{Krav}

Inlämning ska kunna ske direkt från ett system för versionshantering. 

Systemet ska utformas så att det ger alla kategorier av användare en bättre användarupplevelse än 
tidigare system i kategorin, och då i synnerhet i jämförelse med Fire.

Studenter ska ges möjlighet till tät kommunikation med handledaren.

Systemet ska erbjuda handledare eller examinator möjlighet att definiera tekniska och formella specifikationer för en inlämning som kan kontrolleras automatiskt. Inom projektets ramar ska möjligheten till avancerade automatiserade tester undersökas.

Systemet ska vara enkelt att underhålla och vidareutveckla.

\subsection{Frågeställningar}

Inom projektets ramar ska frågor kring anonym rättning, statistik och plagiatkontroll undersökas.

Enligt önskan från projekthandledaren skall tydliga resultatlistor som är kompatibla med LADOK undersökas\footnote{Samtal med Sven-Arne Andreasson den 24 Januari 2012.}.

\section{Disposition}

Rapporten följer gängse struktur: bakgrund, syfte, problemanalys, metod och design, diskussion och slutsats. För att hjälpa läsaren följa med i
de många beslutsprocesserna – beslut som dessutom i många fall beror på varandra – så har vi valt att lägga in korta beslutsavsnitt efter vissa
problembeskrivningar och analyser. I avsnitten beskrivs besluten och deras motivering kort. För utförligare information om beslut hänvisar
författarna till metod- och designavsnitten. I många sammanhang nämns olika produkter och teknologier som kandidatgruppen har använt under analys
eller utveckling av systemet. Dessa samlas tillsammans med sina respektive hemsidor i avsnitt \ref{sec:ordlista}.

\section{Git}
I denna rapport behandlas många frågor som relaterar till versionhanteringssystemet Git. 
För läsare som inte är bekanta med Git erbjuds därför en kort introduktion.

\subsection{Allmänt}
Git är ett distribuerat versionshanteringssystem. Detta innebär att alla användare arbetar med lokala, fullfjädrade \emph{repositorier}. I repositoriet sparas projektets historik som en trädstruktur. Detta kan ses i kontrast mot till exempel SVN, där användaren oftast bara hämtar med en ögonblicksbild – en snapshot – av projektet från en central server.

\subsection{Att inleda arbetet: clone eller init}
För att inleda arbetet med ett projekt kan användaren använda ett av två kommandon: \emph{clone} eller \emph{init}. Med init skapas ett nytt repositorium. Clone kopierar ett befintligt repositorium, inklusive dess kompletta historik.

\begin{Verbatim}
> git init
> git clone git@github.com:user/repository.git
\end{Verbatim}

\subsection{Commit}
När användaren har utfört en arbetsuppgift och vill spara det nuvarande tillståndet i projektet gör han eller hon en \emph{commit}. En commit kan innehålla ett meddelande som beskriver förändringarna som har utförts:

\begin{Verbatim}
> git commit -m "Changed class name"
\end{Verbatim}

\subsection{Push}
När användaren vill dela med sig av sitt arbete kan han eller hon utföra en \emph{push} för att föra över sina ändringar till en annan Git-nod. Eftersom Git är distribuerat kan en Git-nod vara allt från en gemensam server till en kollega som också använder Git. Ett vanligt arbetsflöde är dock att en arbetsgrupp har ett gemensamt huvudrepositorium på en server. Det finns code hosting-företag som erbjuder repositorier som en tjänst.
\begin{Verbatim}
> git push
\end{Verbatim}

\subsection{Pull}
Kommandot \emph{pull} används för att hämta förändringar från ett annat repositorium och sammanfläta dem med det lokala repositoriet. Om användarna har utfört förändringar på samma ställe i projektet kan konflikter uppstå. I de flesta fall kan Git lösa eventuella konflikter på egen hand. Ibland är dock förändringarna så inkompatibla att användaren själv måste lösa dem.

\begin{Verbatim}
> git pull
\end{Verbatim}

\subsection{Grenar: branches}
Grenar, eller \emph{branches}, representerar olika versioner av samma kodbas. Grenar kan användas för att isolera experiment eller nya funktioner från en stabil version av kodbasen. När arbetet är moget nog för att införlivas i huvudgrenen används kommandot \emph{merge} för att sammanfläta grenar.

\section{Ordlista}
\label{sec:ordlista}
\small
\begin{tabular} { | l | p{10cm} | }
\hline
\bf{Ord} & \bf{Betydelse} \\
\hline
Acceptance test	& Test som kontrollerar om koden uppfyller kravspecifikationerna. \\
\hline
Adobe Flash Sockets & Websocket-lösning för ActionScript. \\
\hline
Bare git-repositorium & En git-repositorium utav ett aktivt träd (working tree). \\
\hline
BDD & Behaviour Driven Development. Utvecklingsmetod som bygger på att kodens
beteende beskrivs med hjälp av tester. Beskrivningen skrivs före koden. \\
\hline
Blob & Binary large object.\\
\hline
Branch & En gren i ett repositorium. \\
\hline
Code hosting & Webbtjänst för förvaring av kod. \\
\hline
CommitRequest & Ett anrop som används av webbgränssnittet för att manipulera Git-repositorier. \\
\hline
Constraint & Begränsning i databasen som bara tillåter data som följer en viss specifikation. \\
\hline
Databas-agnostisk & Innebär att den fungerar med alla databaser. \\
\hline
Databas-lager & Logik som implementeras i databasen ligger i databas-lagret. \\
\hline
Databas-schema & Databasstrukturen utan data. \\
\hline
DBMS & Databashanterare likt MySQL. \\
\hline
Diff & Skillnaden mellan två kodbasstycken. \\
\hline
Foreign key & Ett fält i en databastabell som pekar på en annan tabell. \\
\hline
Hash & En heltalsrepresentation av någon sorts data. \\
\hline
Hashfunktion & En algoritm som översätter data till ett heltal. \\
\hline
Hash Fragment & Delen av en URL som kommer efter \#. \\
\hline
LHG & Förkortar "LabHasGroup". Entitet som används i projektet för att koppla laborationsgrupper till laborationer. \\
\hline
Migrationsfil & En fil som används av Rails för att utföra förändringar i en databas. \\
\hline
Modell-lager & Logik som implementeras i modellerna ligger i modell-lagret. \\
\hline
MVC & Model-View-Controller, ett designmönster. \\
\hline
Normaliserad databas & En icke redundant databas som följer normaliseringsreglerna. \\
\hline
Long polling & En förfrågan till extern server. Har servern ingen data att svara med så "hängs" förfrågan. \\
\hline
\end{tabular}

\begin{tabular} { | l | p{10cm} | }
\hline
\bf{Ord} & \bf{Betydelse} \\
\hline
Polling & Att underrätta sig om förändringar i tillstånd genom att göra upprepade förfrågningar. \\
\hline
Git push & Publicering av lokal git till extern eller lokal server. \\
\hline
Remote url & Git-url till extern eller lokal git-server. \\
\hline
Repositorium & Datastruktur som kan innehålla flera versioner av en kodbas sparad på disk. \\
\hline
REST & Representational State Transfer. \\
\hline
Route & Sökväg eller URL till en viss del av en webbapplikation. \\
\hline
Scope & En namnrymd som kan kapsla in saker inuti sig. \\
\hline
Symmetrisk SHA1-signering & Signering av data m.h.a SHA1-algoritmen och ett konstant salt. \\
\hline
Telnet & Nätverksprotokoll för textbaserad kommunikation. \\
\hline
Trigger & En databasrutin som kan köras före, efter eller istället för en operation på en tabell. \\
\hline
Unit test & Test som kollar om en del av koden är redo att användas. \\
\hline
Workers & Trådar som utför jobb i bakgrunden. \\
\hline
Working tree & Filträdet man jobbar med. \\
\hline
\end{tabular}
\normalsize

\section{Länkar till produkter och teknologier}
\small
\begin{tabular} { | l | l | }
\hline
\bf{Produkt/Teknologi} & \bf{Länk} \\
\hline
Backbone.js & http://backbonejs.org/ \\
\hline
Bazaar & http://bazaar.canonical.com/en/ \\
\hline
Beanstalkd & http://kr.github.com/beanstalkd/ \\
\hline
Bundler & http://gembundler.com/ \\
\hline
Capybara & https://github.com/jnicklas/capybara/ \\
\hline 
CoffeeScript & http://coffeescript.org/ \\
\hline
Facebook & http://www.facebook.com/ \\
\hline
Faye & http://faye.jcoglan.com/ \\
\hline
Fire & http://fire.cs.chalmers.se/ \\
\hline
Git & http://git-scm.com/ \\
\hline
Github Inc. & https://github.com/ \\
\hline
Gitorious & http://gitorious.org/ \\
\hline
Grack & https://github.com/schacon/grack \\
\hline
HAML & http://haml.info/ \\
\hline
jQuery & http://jquery.com/ \\
\hline
JUnit & http://www.junit.org/ \\
\hline
Launchpad & https://launchpad.net/ \\
\hline
LESS & http://lesscss.org/ \\
\hline
MiniTest & http://docs.seattlerb.org/minitest/ \\
\hline
MySQL & http://www.mysql.com/ \\
\hline
PostgreSQL & http://www.postgresql.org/ \\
\hline
Redis & http://redis.io/ \\
\hline
Resque & https://github.com/defunkt/resque \\
\hline
RSpec & http://rspec.info/ \\
\hline
Ruby & http://www.ruby-lang.org/ \\
\hline
Ruby on Rails & http://rubyonrails.org/ \\
\hline
RubyGems & http://rubygems.org/ \\
\hline
SASS & http://sass-lang.com/ \\
\hline
SecureFaye & https://github.com/oleander/secure\_faye \\
\hline
Stomp & https://rubygems.org/gems/stomp \\
\hline
TestUnit & http://test-unit.rubyforge.org/ \\
\hline
\end{tabular}
\normalsize
