\section{Köhantering}

Water är en komplex applikation med flera processer som kan bli långdragna eller beräkningsintensiva. För att undvika överbelastning av servern och andra problem bör systemet använda sig av en köhanterare.

\subsection{När behövs en köhanterare?}

\subsubsection{Webbservern}

Ett anrop (HTTP-request) mellan servern och klienten bör inte ta längre tid än 400 ms, allt där över kan stalla webbservern vilket kan leda till att andra webbförfrågningar får vänta. En stallad server kan få webbtjänsten att kännas långsam, vilket kan uppfattas negativt av användaren.
En lösning på problemet är att göra ett HTTP-anrop som initierar ett asynkront jobb för de tyngre operationerna. Jobbet läggs på kö i stället för att behandlas på plats för att sedan bearbetas av en extern instans som lite längre fram i tiden. När processen är klar så svarar klienten med data, till exempel via Websocket-protokollet.

\subsubsection{Tunga, blockerande processer}

Läsningar och skrivningar till disk, så kallade I/O-operationer, är blockande. En blockerande operation kan låsa OS-trådar och ta resurser från mer högprioriterade processer såsom webbserver-förfrågningar.
Vid få I/O-operationer är inverkan på systemet relativt litet, men kan vid ökad mängd ha stor, negativ inverkan.
Lösningen är att låta systemet jobba av en mängd operationer under en begränsad tid. På så vis kan vi försäkra om oss att de tyngre, blockande operationer inte gå överstyr och på så vis tar ner vårt system.

\subsection{Val av köhanterare}

Det finns en del köhanterare att välja mellan. De skiljer en del mellan dem, men principen är alltid den samma. Gitorious använder sig från start av köhanteraren beanstalkd.
De två köhanterarna som analyserats för Waters bruk är:

\begin{itemize}
  \item Resque
  \item Beanstalkd
\end{itemize}

\subsection{Resque}

Resque är en köhanterare som bygger på key value-databasen Redis. Plattformen
är byggd av Github Inc. i språket Ruby och har en medföljande
inställningspanel. Administrationspanelen gör det möjligt att starta om
fallerade jobb, se statistik och hantera prioriteringar. Redis är ett etablerat
verktyg som används av ett flertal stora aktörer (Redis), bland annat den sociala utvecklingsplatformen Github.

Ett problem med Resque är dock att den använder polling för att hämta data från
Redis. Polling-problematiken gör att informationen som hämtas av en \emph{worker} vid
en given tidpunk kan vara utdaterad. Då Resque är byggd i Ruby är bryggan
mellan mjukvaran och Water väldigt liten. Implementeringen sker enkelt med
hjälp av en rad kod:

\begin{verbatim}
  mount Resque::Server, :at => "/resque"
\end{verbatim}

\subsection{Beanstalkd}
Beanstalkd är en realtidsköhanterare byggd i språket C skriven för att hantera asynkrona jobb för den sociala plattformen Facebook (Beanstalk 2007). Beanstalkd har ett lätt fotavtryck, har möjlighet att beta av jobb i realtid men saknar en administrationspanel.
Beanstalkd har ett flertal bibliotek till Ruby som gör implementeringen enkel. Ett utav dem är Stomp som finns färdigimplementerat i Gitorious.
