\section{Tekniska frågor kring validering}

Validering av inlämningar kan ske på olika nivåer – enkel kontroll av formalia, kompilering och testning. 
Att implementera kontroll av formalia, som att vissa filer ska finnas i inlämningen eller att filerna namnges på speciella sätt, är inte svårt. Mer avancerade former av validering ger dock upphov till större tekniska utmaningar.

\subsection{Kompilering}
\label{section:kompilering}

Att en inlämning inte går att kompilera torde i de flesta fall leda till att den blir underkänd. Om systemet kunde testa detta automatiskt skulle det kunna lätta handledarnas arbetsbörda.
För kompilering krävs dock att lämpliga utvecklingsmiljöer finns tillgängliga i systemet. Eventuella externa paket som används måste också läggas till i systemet eller skickas med i inlämningen, om utvecklingsmiljön tillåter det.
Att stöda en stor mängd språk blir arbetsintensivt, speciellt om det uppstår problem kring olika versioner av språk och paket.
Ett realistiskt första steg vore att bara stöda Java. På Chalmers används Java som programmeringsspråk bland annat i kurser som Objekorienterad Programmering, Parallellprogrammering, Datastrukturer och algoritmer och Webbprogrammering (Chalmers, 2012). Inom branschen i stort har Java länge varit det ett av de största språken – 2012 ligger det på placering två efter C (TIOBE Software, 2012).

Kompilering är tidskrävande och bör därför hanteras asynkront av någon form av kösystem.

\subsection{Unit/Acceptance tester}
Ett ytterligare verktyg för handledare samt examinatorer skulle vara att man i systemet kunde definiera \emph{unit-} och \emph{ acceptancetester}. Detta förutsätter att det går att kompilera kod och lider därför av samma problem som de som beskrivs i avsnitt \ref{section:kompilering}.
\subsubsection{Exekvering av studenters kod}
Att exekvera studenters kod på servern är också förenat med säkerhetsrisker.
För att säkert köra extern kod så skulle systemet behöva upprätta sandboxade miljöer. Helst bör även all beräkningstung validering hanteras asynkront av en köhanterare och ett antal worksers. Ett sådant system är komplicerat och endast färdiga lösningar såsom Jenkins (Jenkins, 2012) eller Bamboo (Atlassian, 2012) är tänkbara.

\subsection{Asynkron validering och deadlines}

Om inlämningar valideras asynkront av systemet kan det uppstå situationer där studenter lämnar in strax innan deadline och får meddelande om retur när tiden har gått ut. Hur dessa situationer ska hanteras måste utredas, förslagsvis i samråd med handledare och lärare.
