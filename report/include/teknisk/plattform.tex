\section{Plattform}
För de resterande delarna av systemet fanns det flera olika tillvägagångssätt. Det går att exempelvis bygga backend, logik och webbgränssnitt ifrån grunden. Gruppen befarade att detta skulle uppta fler arbetstimmar än vad som fanns till förfogande för projektet.

Ett annat alternativ var att använda sig av en befintligt modul för att visa versionshanterade projekt på webben och bygga resten av webbgränssnittet och backend själva.

En tredje möjlighet som man behövde ta ställning till var att använda befintliga lösningar för så kallad “source code hosting”. Dessa innehåller i regel webbgränssnitt och backend för hantering och presentation av versionshanterade projekt. Systemet skulle byggas ut med logik för inlämningsuppgifter. Tyvärr finns det väldigt få lösningar som bygger på öppen källkod. Ofta är endast delar av systemen öppna. De vanligaste helt öppna exemplen är Gitorious och Launchpad.

Launchpad stöder endast Bazaar medan Gitorious bygger på Git. Därmed finns en koppling mellan valet av source code hosting-lösning och versionshanterare.

Gitorious är byggt i språket Ruby med ramverket Ruby on Rails. Projektgruppen har erfarenhet av ramverket sen tidigare. 

Launchpad är skrivet i Python och Javascript. Systemet kräver i skrivande stund Ubuntu för full funktionalitet (Launchpad Blog 2009). Systemet är mycket brett och innehåller bland annat moduler för översättning av program, paketering, bugghantering, kollaborativ utformning av kravspecifikationer och system för dynamisk konfiguration av servern.

Oberoende av vilken plattform som skulle användas så fanns det skäl att skala bort den funktionalitet som inte behövs för projektets ändamål. Detta talade emot Launchpad, som innehåller en stor mängd överskottsfunktionalitet.