\section{Licensiering}
Licensiering är i sig ingen teknisk fråga, men för projektet är licensiering starkt sammakopplat med valet av plattform. 

Både Gitorious och Launchpad är licensierade med GNU Affero General Public License (Gitorious, 2012; Launchpad, 2010) hädanefter kallat GNU Affero GPL. I sin artikel om hur licensen uppkom berättar Vaughan-Nichols (2007) hur licensen ursprungligen kommer ifrån GNU General Public License vilken är en så kallad “Share-alike/Copyleft”-licens.

Konceptet “Copyleft” är resultatet av en  ordlek på den juridiska termen “Copyright”. Copyright nyttjas för att garantera den ursprungliga tillverkaren exklusiva rättigheter till sin produkt. Copyleft, vilket felaktigt kan uppfattas som en motsägelse till Copyright, använder sig utav upphovsrätten för att garantera alla rätten att använda, re-distribuera samt modifiera produkten - så länge som de använder sig av samma licensiering som den ursprungliga produkten (GNU, 2012).

En produkt som använder sig av GNU Affero GPL måste låta källkoden vara tillgänglig för alla användare av produkten. För webbapplikationer som Gitorious och Launchpad innebär detta att källkoden måste vara tillgänglig kostnadsfritt på en server för samtliga användare av applikationen. 
Licenstexten måste göras tillgänglig i samtliga källkodsfiler i systemet. Antingen finns licenstexten i varje fil, eller så finns en central fil som innehåller licenstexten som alla andra filer refererar till (GNU, 2007).

Sammanfattningsvis innebär detta för kandidatarbetet att oavsett vilken plattform som projektet väljer måste all källkod alltid vara öppen, att vidareutvecklingar av Water alltid måste vara licensierade av GNU Affero General Public License och licensetexten måste alltid gå att finna i eller med hjälp av samtliga filer.

\begin{flushright}
  
  \textbf{Beslut}
  
  Gitorious valdes som grundsystem eftersom kompetens inom Ruby och Ruby on Rails fanns i gruppen och eftersom Launchpad innehåller mycket funktionalitet som inte behövdes i projektet. Eftersom Gitorious är baserad på Git, kommer Git också vara valet av den distribuerade versionshanteringssystemet. Projektet kommer därmed att ligga under GNU Affero GPL-licensen.  
  
\end{flushright}
