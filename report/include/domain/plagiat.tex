\section{Plagiatkontroll}
På Chalmers upprätthålls den akademiska hederligheten. För inlämningsuppgifter, speciellt programmeringsuppgifter, är det vanligt förekommande att man får samarbeta i någon form. Beroende på inlämningsuppgiftens instruktioner är det ibland godkänt att samarbeta mellan grupper, medan det ibland är av vikt att gruppen lämnar in en unik lösning (Chalmers tekniska högskola, 2009). I det senare fallet hade det underlättat för lärarna om automatiskt plagiatkontroll implementerades direkt i inlämningssystemet.

Urkund är ett system för automatisk plagiatkontroll. Texter som behandlas av Urkund jämförs med en bred databas som består av tidigare verk. I dagsläget använder lärare sig utav e-post för att skicka material till Urkund vid misstanke om plagiat. Då Urkund erbjuder ett API till sina kunder (URKUND, 2012), skulle denna process kunna göras betydligt smidigare genom att integrera funktionaliteten direkt i Water.

Ytterligare skulle en intern kontroll av tidigare inlämningar kunna implementeras. För att detta skulle vara möjligt måste samtliga inlämningar som laddas upp till Water behöva sparas i databasen. Därefter skulle en lämplig algoritm för filjämförelse behöva implementeras.

\begin{flushright}
  
  \textbf{Beslut}
  
  Plagiatkontroll nedprioriteras och implementeras inte.
  
\end{flushright}
