\section{Inlämningskanaler}

Det nya systemet ska vara anpassat för användare med olika kunskapsnivåer. En konsekvens av detta är att inlämning via versionshanteringsklient inte kan vara det enda tillvägagångssättet. Studenter vid olika program och vid olika skeden av sin utbildning kan ha varierad eller ingen erfarenhet av versionhantering. Det är därför viktigt att erbjuda flera kanaler.

\subsection{Versionshanterare}
Versionshanterare som inlämningskanal behandlas i avsnitt \ref{sec:fragor-git}, sida \pageref{sec:fragor-git}.

\subsection{Webbgränssnitt}
Uppladdning genom webbgränssnitt är den lösning som används i Fire. Metoden är robust och bekant för de flesta. Projektgruppen anser dock att lösningen lider av ett antal begränsningar. Det enda sättet att ladda upp flera filer på samma gång är med arkivfiler i tar.gz-format (Gedell 2006a). Detta är också det enda tillgängliga sättet att ladda upp mappstrukturer. Det går heller inte att bläddra i innehållet i filer eller tar.gz-arkiv, vilket leder till att studenter ofta känner sig tvungna att ladda ner en hel inlämning igen för att känna sig säkra på att innehållet är korrekt.

Water bör erbjuda ett mer flexibelt sätt att ladda upp filer. Uppladdning av flera filer på samma gång bör vara möjligt. Det bör också gå att skapa mappar och ladda upp filer till olika ställen i en mappstruktur. För att detta ska vara möjligt bör Water möjliggöra navigering i filstrukturer.

\subsection{E-post}
Ytterligare en potentiell metod för inlämning är via e-post. Studenter skulle då kunna skicka in inlämningar som bifogade filer. För att knyta en inlämning till en specifik uppgift inom en viss kurs så skulle systemet kunna generera unika identifikationskoder som studenten skickar med i mailet. Förslagsvis skulle man vid registrering för en kurs i systemet få ett mail per uppgift som innehåller all information som behövs för att kunna utföra laborationen. Studenten skulle då kunna lämna in till en specifik uppgift genom att svara på mailet som är knutet till den uppgiften.

Som vi såg det skulle logiken för mailinlämningar riskera att bli väldigt invecklad. Om syftet med att utöka antalet inlämningskanaler är att höja användarvänligheten, så finns det ingen anledning att lägga till en kanal där inlämningsförfarandet är onödigt komplicerat.

\begin{flushright}
  \textbf{Beslut}
  
  Inlämningskanalerna begränsas till versionshanterare och webbgränssnitt för att undvika komplexiteten i att implementera inlämning via e-post.
\end{flushright}

