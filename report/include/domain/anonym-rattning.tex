\section{Anonym rättning}

På CTH har anonym rättning införts för alla salstentamina (Chalmers tekniska högskola, 2008). För att förbättra studenternas rättssäkerhet valde kandidatgruppen att utreda om anonym rättning kunde införas även för laborationer.

Fördelar med anonym rättning är att studenter är garanterade en rättvis rättning där rättarens personliga åsikter inte kan påverka resultatet, anonym rättning innebär också att studenter kan argumentera med sina lärare om kursen och dess innehåll utan att oroa sig för negativa effekter vid bedömning.

Ett enkelt sätt att implementera detta på är att ersätta studenternas namn med sifferkoder eller liknande i vyerna. Då är det sparat i databasen vem det är och en administratör kan lämna ut uppgifterna om en elev eller grupp om nödvändigt.

Anonym rättning lämpar sig inte nödvändigtvis för alla sorters laborationer och så länge det inte finns något policybeslut om det från universitetet bör användandet av funktionen vara valbart för var kurs.

\begin{flushright}

  \textbf{Beslut}

  Då anonym rättning inte är en kritisk del av projektet har det blivit nedprioriterat.
Eftersom implementetationen är relativt enkel kommer den här funktionen troligtvis dyka upp i en senare version av mjukvaran.
\end{flushright}
