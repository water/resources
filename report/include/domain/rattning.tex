\section{Rättning}

I Fire finns ett antal uppskattade funktioner för handledare. Bland annat kan handledaren ladda ner samtliga inlämningar som den ska rätta i ett tar.gz-arkiv (Gedell 2006b). Funktioner som dessa bör finnas även i Water.

Fire har dock några brister i rättningsfunktionen.

Handledare upplever ibland att det är svårt att se exakt vad som har ändrats mellan två inlämningar\footnote{Intervju med Oskar Utbult den 2 mars 2012}. Genom att som i Water basera inlämningssystemet på versionshantering är det lätt att ta fram skillnaden mellan två tillstånd – en såkallad \emph{diff}. En handledare som använder sig av ett versionshanteringssystem för att hämta inlämningar kan enkelt skapa en \emph{diff} själv. Water bör erbjuda visualisering av \emph{diffar} i webbgränssnittet.

Fire låter användaren hämta rena textfiler via webbgränssnittet, men kan inte hantera mappstrukturer. Studenter som vill lämna in projekt med annat än en endimensionell mappstruktur måste använda arkivfiler för att göra detta. Detta innebär även att handledarna måste ladda ner projekten som arkivfiler, istället för att förhandsgranska den i webgränssnittet.

Många inlämningar vore möjliga att rätta direkt i ett webbgränssnitt om det erbjöd navigering i mappstrukturer och rendering av källkodsfiler. Detta skulle göra handledarnas jobb mer flexibelt. Water bör erbjuda denna funktionalitet.
