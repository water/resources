\section{Inlämningens livscykel}

Grundantagandet i Water är att en inlämning i sin enklaste form är en pekare till ett visst tillstånd i ett repositorium – en commit. Inlämningen går igenom ett antal tillstånd innan uppgiften kan godkännas. När denna livscykel ska utformas ställs krav som enkelhet, flexibilitet och tydlighet mot varandra.

\subsection{När ska inlämningskanalerna öppnas och stängas?}
Grundidén är att inlämningskanalerna stängs när deadline passerat eller när en inlämning har gjorts för att undvika att rättaren arbetar med en inlämning som inte längre är aktuell. Kanalerna öppnas igen om inlämningen är underkänd.

Ur studenternas perspektiv är det dock önskvärt att kunna göra ändringar i en inlämning så länge som möjligt. En typisk situation är att filer fattas eller att något uppenbart fel uppdagas efter att inlämning har skett. Det förefaller onödigt att låta studenterna vänta på att en rättare underkänner inlämningen innan de kan rätta felet.

En kompromiss är att tillåta ändringar i en inlämning tills rättaren har börjat arbeta med den. 

\subsection{Uppdatering av inlämningar}\label{sec:uppdatering-inlamningar}
Teoretiskt vore det möjligt att se ändringar av inlämningar som nya inlämningar. I så fall skulle studenterna ha rätt att göra ett obegränsat antal inlämningar tills dess att rättaren inleder sitt arbete. Detta leder dock till att det kan finnas föräldralösa inlämningar i systemet som aldrig kommer att behandlas, vilket gör hanteringen av dem svårare.

För att undvika detta är det möjligt att se ändringar i inlämningar som uppdateringar av en befintlig inlämning. Vid uppdatering riktas pekaren i inlämningen till ett nytt tillstånd i inlämningsrepositoriet. Det öppnar även för enklare hantering av till exempel deadlines, eftersom till exempel den första deadlinen kan relateras direkt till den första inlämningen.

\subsection{När inlämningen har rättats}
Om en inlämning blir godkänd bör det inte längre vara möjligt att lämna in fler inlämningar. Om inlämningen däremot är icke-godkänd, och chans för ytterligare inlämning finns, bör inlämningskanalerna öppnas igen.

\begin{flushright}

  \textbf{Beslut}
  
  \nopagebreak
  
  En inlämning kan uppdateras så att den pekar på ett nytt tillstånd i ett repositorium tills handledaren har börjat rätta den. Blir inlämningen underkänd kan en ny inlämning skapas. Blir inlämningen godkänd är det inte längre möjligt att lämna in.
\end{flushright}
