\section{Statistik}

Att ha tillgång till statistik från tidigare år hjälper examinatorer att successivt utforma innehållet av laborationer. Till exempel kan en laboration vara för omfattande och då ta för lång tid, vilket resulterar i att de flesta inlämingar kommer in sent, eller inte alls. Det kan också vara av intresse för examinatorn att se hur stor andel av studenterna klarade av en viss laboration.  

Detta kan tas fram genom att implementera attribut i databasen, med vilka statistiken kan tas fram, till exempel ett attribut som innehåller tidpunkten då en grupp gjorde en inlämning.

Datan kan sammaställas i en tabell, med vilken man lätt kan jämföra olika årgångar.
