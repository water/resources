\section{Tillståndshantering - vilken entitet är det som blir godkänd?}

För att implementera inlämningens livscykel behövs ett antal tillstånd och strategier för att hantera de olika stegen.

För betygssättning behövs tillstånd för godkända och underkända inlämningar. För att kunna implementera uppdatering av inlämningar, vilket behandlas i avsnitt \ref{sec:uppdatering-inlamningar}, behövs skilda tillstånd för inlämningar som väntar på bearbetning av handledare och inlämningar där arbetet har inletts.

Vid första anblick är det oklart vilken entitet i modellen som ska hantera dessa tillstånd. Tydligt är dock att handledarna i första hand kommer att behandla inlämningar för att kunna sätta ett betyg på en inlämningsuppgift. Den naiva lösningen är alltså att låta inlämningsentiteterna hantera tillstånd.

Det är på samma gång tydligt att det är grupper som får hela inlämningsuppgifter godkända, vilket talar för att tillståndet istället lagras i en entitet som kopplar grupper till uppgifter.

\subsection{Tillståndshantering i flera entiteter leder till redundans}
Ytterligare en lösning är att placera tillstånden där de tycks passa bäst – både i inlämningentiteten och i entiteten som kopplar grupper till laborationer. Detta skapar dock redundans i modellen eftersom informationen finns tillgänglig även om den bara sparas i den ena entiteten. Till exempel betyder en godkänd inlämning att även laborationen är godkänd. En godkänd laboration betyder att den sista inlämningen är godkänd och de övriga är icke-godkända eller icke rättade, beroende på hur modellen är beskaffad.

Redundans bör undvikas av de anledningar som diskuteras i kapitel \ref{sec:databasen}.

\subsection{Tillståndshantering i endast en entitet kan leda till oklara tillstånd}
Om tillståndshanteringen endast placeras i inlämningsentiteten blir det lätt att hämta information om enskilda inlämningar. Däremot måste alla inlämningsentiteter för en grupp och en inlämningsuppigft hämtas och kontrolleras när systemet ska beräkna hela inlämningsuppgiftens tillstånd.

Om tillståndshanteringen istället placeras i entiteten som kopplar grupper till inlämningsuppgifter är det lätt att hämta information om gruppens tillstånd i förhållande till inlämningsuppgiften. Det kan i gengäld bli svårare att ta reda på tillståndet för en enskild inlämning, beroende på hur modellen är uppbyggd. Problem uppstår om systemet tillåter föräldralösa inlämningar, vilket beskrivs i avsnitt \ref{sec:uppdatering-inlamningar}, och gruppen har gjort två eller flera inlämningar och den sista inlämningen är icke-godkänd. I detta fall är det omöjligt att veta om inlämningarna före den icke-godkända inlämningen är icke-godkända eller föräldralösa. Det går att betrakta alla inlämningar före en icke-godkänd inlämning som icke-godkända, men det leder till att inlämningar som aldrig har behandlats av en handledare ser ut att ha blivit behandlade.

Om systemet inte tillåter föräldralösa inlämningar går det att entydigt bestämma enskilda inlämningars tillstånd utifrån gruppens tillstånd i relation till inlämningsuppgiften.

\begin{flushright}

  \textbf{Beslut}

  Tillstånd hanteras i entiteten LabHasGroup, som beskrivs i modellavsnittet \ref{sec:modell-labhasgroup}. Entiteten representerar grupper i relation till inlämningsuppgifter. Enligt tidigare beslut tillåts inte föräldralösa inlämningar, vilket gör att enskilda inlämningars tillstånd kan entydigt bestämmas.
\end{flushright}
