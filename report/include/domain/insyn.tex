\section{Insyn i rättningsprocessen}

Från studenternas sida finns det en önskan om större insyn i rättningsprocessen.

En möjlighet är att upprätta ett kösystem. Samtliga inlämningar i hela kursen läggs i en rättningskö där de prioriteras efter vilken uppgift inlämningen gäller, inlämningstid, om inlämningen är svar på en retur och andra liknande faktorer. En ledig handledare skulle vara tvungen att rätta den inlämning som står längst fram i kön, i detta fall skulle studenterna få väldigt exakta uppgifter om sin köplats. Problemet med detta system är att reglerna för kösystemet skulle bli tämligen rigida och att de nödvändigtvis inte skulle passa för alla sorters kurser.

En annan möjlighet är att ge handledarna den information som de behöver för att själva göra prioriteringarna utan att bygga en tvingande kö i systemet. På detta sätt är systemet mer flexibelt men det skulle bli svårt att ge studenterna tydliga indikationer om hur rättningen fortskrider.

En kompromiss är att inlämningarna delas in i mindre prioritetsköer beroende på vilken laboration de hör till. Handledarna väljer vilken laboration de vill rätta, men är bundna till prioritetsköerna för den valda laborationen. 

\begin{flushright}

  \textbf{Beslut}

  Inga rigida prioritetsköer implementeras för rättning. Rättarna får göra prioriteringarna själva. Detta eftersom behoven i olika kurser kan variera på sätt som projektgruppen inte kan förutse.
\end{flushright}
