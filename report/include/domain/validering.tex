\section{Validering av inlämningar}

Laborationer har i allmänhet krav som måste uppfyllas för att de skall bli godkända. Typiska exempel på detta kan vara att inlämningen skall bestå av filer med speciella namn, ha en viss typ av filformat, maximalt antal tecken i radbredd eller gå igenom de tester som bifogats med laborationsbeskrivningen.

I det nuvarande systemet måste handledare manuellt neka inlämningar som inte uppfyller dessa krav. Handledarna får spendera tid på att avvisa laborationer som inte är giltiga då studenterna har slarvat med innehållet, men kanske har en korrekt lösning. Studenterna i sin tur förlorar ett av sina inlämningsförsök och blir först medvetna om detta efter att handledaren fått tid att titta på den.

En lösning till detta vore att i det nya systemet erbjuda automatiska valideringar vid varje inlämning. Dessa skulle kunna, så fort testerna är klara, meddela studenten om inlämningen uppfyller de grundläggande kraven - så denne får en omedelbar chans att rätta till felaktigheterna. Som oftast har studenterna endast ett fixerat antal inlämningar på sig innan en laboration måste vara godkänd. Om en inlämning som inte passerar testerna skall minska antalet kvarvarande försök eller ej är något som skulle behövas ta ställning till.

Vad som skulle ingå i valideringen av en inlämning skulle behöva konfigureras efter specifikationerna för varje laboration. 
