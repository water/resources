\section{Kommunikation}

Ett syfte med Water är att förbättra kommunikationen mellan studenter och personal. Dessa förbättringar kan ske på olika nivåer.

Ett system som Water skulle kunna användas som allmän informationskanal, till exempel för meddelanden som handlar om kursval och dylikt,  eftersom det har tillgång till samtliga studenter som är registrerade på en kurs. Alla kurser på en institution kommer dock inte att använda Water. Därför är det bättre att använda andra verktyg för allmän kommunikation med studenterna. Fokus i Water ligger istället på att förbättra den kommunikation som sker mellan studenter och handledare i anknytning till inlämningsuppgifter.

\subsection{Kommunikation kring inlämningsuppgifter}
Som tidigare nämnt kan handledare i Fire inkludera kommentarer när de graderar en inlämningsuppgift. Studenterna kan endast svara med en ny inlämning. Studenterna riskerar att missuppfatta återkopplingen, vilket kan leda till att även den korrigerande inlämningen förkastas onödan. För att råda bot på detta vill projektgruppen att Water  ska erbjuda en dialog mellan handledare och student.

Detta kan implementeras genom ett kommentarssystem. Systemet låter handledare kommentera inlämningar. Handledarens kommentarer presenteras sedan tillsammans med koden i webbgränssnittet. Studenterna kan därefter svara på handledarens kommentarer med egna kommentarer.

Ytterligare en möjlighet är att erbjuda så kallade inline-kommentarer, det vill säga kommentarer som är knutna till specifika rader i inlämnade filer. Detta skulle göra det möjligt för handledarna att ge återkoppling på väldigt specifika problem, istället för att föröka referera till dem i löpande text i en kommentar till inlämningen.

Alla parter bör bli underrättade när nya kommentarer har kommit in.
