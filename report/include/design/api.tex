\section{CommitRequest: Ett API för manipulering av repositorier via webbklienten}
Webbgränssnittet utför operationer på Gitrepositorier genom att göra anrop till webbservern, som i sin tur förmedlar anropen till köhanteraren. Projektgruppen har valt att kalla dessa anrop för CommitRequest.
För anropen har projektgruppen definierat ett API. Samma dataformat används för att skicka vidare anropen från webbservern till köhanteraren.
Operationerna som stöds är:

\begin{itemize}
  
  \item add – Lägg till filer
  \item rm – Ta bort filer och mappar
  \item mkdir – Skapa en mapp
  \item mv – Flytta filer
  
\end{itemize}

Alla operationer kräver ett repositorie-id samt namnet på den branch som ska användas. Alla operationer stöder också att ett commit-meddelande skickas med anropet. Ett tomt commit-meddelande leder till att servern genererar ett automatiskt meddelande. Add och mv kräver en ursprungssökväg och en destinationssökväg. Rm och mkdir kräver bara sökvägar.
Se bilaga 1 för en exakt specifikation.