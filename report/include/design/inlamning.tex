\section{Inlämning}
\subsection{Inlämning via Git}
\subsection{Inlämning via webbgränssnittet}
\subsection{Inlämningens livscykel}

För att implementera inlämningsuppgiftens livscykel har tillståndshantering implementerats i LabHasGroup-entiteten. LabHasGroup representerar kopplingen mellan en laboration och en grupp.
Tillstånden beskrivs som följer:

\subsubsection{Initialized}

När en LabHasGroup skapas sätts objetet i tillståndet initalized. I detta tillstånd finns inga Submissions kopplade till entiteten.

\subsubsection{Pending}

När gruppen gör en inlämning skapas en Submission och LabHasGroup-entiteten övergår till tillståndet pending. Pending betyder att en inlämning är gjord, men att ingen handledare granskat laborationen.
Om en LabHasGroup befinner sig i pending kan dess senaste Submission uppdateras så att den pekar på ett nytt tillstånd i repositoriet.

\subsubsection{Reviewing}

Så snart en handledare har påbörjat granskningen en laboration övergår LabHasGroup-entiteten till tillståndet reviewing. I detta tillstånd är det inte möjligt att göra flera inlämningar

\subsubsection{Rejected och Accepted}

Efter granskning kan handledaren godkänna eller förkasta inlämningen. Om inlämningen är godkänd går LabHasGroup-entiteten över i tillståndet accepted. Inga fler inlämningar tas emot. Om inlämningen förkastas går entiteten över i tillståndet rejected. I rejected-tillståndet är det möjligt att göra en ny inlämning, varpå tillståndet övergår i pending.