\section{Hantering av roller}

Varje användare i systemet är en User, och olika Users kan ha olika ansvarsområden. I Water har fyra typer av användare modellerats:

1. Student som ska simulera en student som kan bland annat göra inlämningar och registera sig på kurser.
2. Assistant som ska simulera en handledare som kan bland annat rätta labbar.
3. Examiner som ska simulera en examinator, som kan bland annat kan utnämna handledare.
4. Administrator som ska simulera en administratör, som kan bland annat skapa nya kurser.  
 
För att användare ska ha tillgång till olika funktionaliteter, beroende på vilken typ av användare de är har roller implementerats. Funktionalitet för de olika rollerna befinner sig i en \emph{scope}, och en användare har bara tillgång till den funktionalitet som dess roll tillhandager. Om en användare är en adminstratör kan den skapa nya kurser, men en användare som är en student tillåts inte göra det.

\subsection{Rollernas påverkan på routes}
Användarfunktionaliteten befinner sig inom en \emph{scope} med roller. Detta ger upphov till estetiska, och intuitivt förstådda URL:er. Till exempel kan kurser visas upp för en student genom: |/student/courses|. Om studenten också är en Assistant, används en annan route för att visa relevant information för det sammahanget.

Sidor som till exempel visar en mall för redigering av User-information ska inte vara \emph{scopade}, eftersom de inte tillhör någon specifik roll-kontext.
