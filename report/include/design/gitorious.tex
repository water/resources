\section{Delar av Gitorious som används i Water}
Stora delar av Gitorious har tagits bort under anpassningen till Water.

Den viktigaste delarna som används är:
\begin{itemize}
  \item Repository-modellen och tillhörande interaktion med Git
  \item Rendering av träd och BLOB:ar
  \item User-modellen samt auktorisering och sessionshantering.
\end{itemize}

Repositorymodellen erbjuder metoder som låter systemet interagera med Git-repositoriet som modellen representerar. Den gör det enkelt att hämta fysiska sökvägar till repositoriet och information om till exempel commits.

En av de viktigaste delarna är rendering av filträd och objekt. Waters webbgränsnitt hämtar både träd och objekt som HTML-fragment från controllers som härstammar från Gitorious. 
Gitorious känner av olika typer av objekt och kan avgöra om de ska renderas som källkod eller till exempel som bildtaggar. Trädvyerna går att hämta för olika \emph{branches} och commits, och full navigering i trädet är möjlig.

I framtiden bör Water använda auktorisering via Chalmers, till exempel med hjälp CID. Under tiden används dock vanliga användarkonton från Gitorious, med tillhörande auktorisering.
